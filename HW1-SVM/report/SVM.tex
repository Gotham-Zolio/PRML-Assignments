% !Tex program = pdflatex

\documentclass[UTF8]{ctexart}
\CTEXsetup[format={\Large\bfseries}]{section}
\usepackage{amsmath}
\usepackage{ctex}
\usepackage{array}
\usepackage{ulem}
\usepackage{graphicx}
\usepackage{booktabs}
\usepackage{listings}
\usepackage{xcolor}
\usepackage{geometry}
\usepackage{multirow}
\usepackage{subfig}
\usepackage{float}
\usepackage{multicol}
\usepackage{multirow}
\usepackage{pgfplots}
\usepackage{diagbox}
\usepackage{indentfirst}
\setlength {\parindent} {2em} 
% small vertical space between paragraphs for readability
\setlength{\parskip}{0.6em}
\usepackage{makecell}
\geometry{papersize={21cm,29.7cm}}
\geometry{left=2.54cm,right=2.54cm,top=3.18cm,bottom=3.18cm}
\usepackage{fancyhdr}
\pagestyle{fancy}
\lhead{\today}
\chead{}
\rhead{2023012274}
\lfoot{清华大学}
\cfoot{\thepage}
\rfoot{模式识别与机器学习}
\renewcommand{\headrulewidth}{0.4pt}
\renewcommand{\headwidth}{\textwidth}
\renewcommand{\footrulewidth}{0pt}
\usepackage{bm}
% listings style for code snippets
\lstset{
    backgroundcolor=\color{white},
    basicstyle=\ttfamily\small,
    breaklines=true,
    framesep=4pt,
    frame=single,
    rulecolor=\color{gray!40},
    keywordstyle=\color{blue},
    commentstyle=\color{gray!60}\itshape,
    showstringspaces=false,
}

\begin{document}
\begin{titlepage}
     \begin{center}
        \quad \\
        \quad \\
        \quad \\
        \quad \\
        \quad \\
        \quad \\
        \quad \\
        \quad \\ 
        \quad \\ 
        {\kaishu \fontsize{30}{15}\selectfont 《模式识别与机器学习》}\\
        \quad \\
        \quad \\
        {\kaishu \fontsize{30}{15}\selectfont 实验:SVM 图像分类}\\

        \end{center}
        \vskip 8cm

        \begin{center}
        \begin{large}
        \begin{tabular}{cc}
        院\qquad 系:& ~~~~~~~~自动化系~~~~~~~~      \\
        \cline{2-2}\\
        班\qquad 级:& 自31班   \\
        \cline{2-2}\\
        学生姓名:& 苟左    \\
        \cline{2-2}\\
        学\qquad 号:&2023012274   \\
        \cline{2-2}
          \end{tabular}
        \end{large}
      \end{center}

\end{titlepage}
\newpage

\section{实验目的}
\begin{enumerate}
    \item 理解支持向量机(SVM) 的基本原理与超参数含义(尤其是惩罚参数 $C$ 和核函数参数)。
    \item 在给定的人脸数据集上实现基于属性与基于原始像素的 SVM 分类,并比较两种特征表示的性能差异。
    \item 学习如何使用标准化、降维与核函数调参来提升分类器的泛化能力。
\end{enumerate}

\section{实验任务与方法}

\subsection{数据与任务}
本实验使用提供的 \texttt{celeba} 子集,分为训练/测试两个子集(由 \texttt{partition.txt} 指定)。数据集包含:
\begin{itemize}
    \item 训练样本:250 张人脸图像
    \item 测试样本:50 张人脸图像
    \item 分类任务:10 类人脸识别(标签映射见 \texttt{dataset.py} 中 \texttt{label\_mapping})
    \item 特征:40 维人脸属性特征 + 原始像素($218 \times 178 \times 3 = 116412$ 维)
\end{itemize}

\subsection{实现要点(对应 \texttt{svm.py} 中 TODO)}
\begin{enumerate}
    \item \textbf{特征选取}:
    \begin{itemize}
        \item 属性特征(\texttt{X\_attrs}):直接使用 40 维人脸属性作为低维输入
        \item 原始图片(\texttt{X\_imgs}):将 $[H, W, 3]$ 展平成一维向量,或先做降维/特征提取
    \end{itemize}
    
    \item \textbf{数据预处理}:使用 \texttt{StandardScaler} 进行零均值单位方差标准化,消除不同特征维度的量纲差异
    
    \item \textbf{模型训练}:使用 sklearn 的 \texttt{SVC} 线性核,通过网格搜索优化惩罚参数 $C$
\end{enumerate}

\section{实验实现与代码片段}

\subsection{基础 SVM 分类器实现(svm.py)}
下面摘录 \texttt{code/svm.py} 中关键实现。TODO 部分:特征选择和 SVM 训练。

\begin{lstlisting}[language=Python, caption=svm.py 核心代码片段]
# 构建数据集
data = Dataset("../celeba")
X_imgs_train, X_attrs_train, Y_train = data.get_train_data()
X_imgs_test, X_attrs_test, Y_test = data.get_test_data()

# TODO 1: 特征选择
# 方案1:选择属性特征作为分类依据
X_train = X_attrs_train
X_test = X_attrs_test
# 方案2:选择原始图片作为分类依据(需要展平)
X_train = X_imgs_train.reshape(X_imgs_train.shape[0], -1)
X_test = X_imgs_test.reshape(X_imgs_test.shape[0], -1)

# 标准化训练集和测试集
sc = StandardScaler()
sc.fit(X_train)
X_train_std = sc.transform(X_train)
X_test_std  = sc.transform(X_test)

# TODO 2: 训练支持向量机(需要选择合适的 C 值)
svm = SVC(kernel="linear", C=1.0)
svm.fit(X_train_std, Y_train)

# 预测与评估
Y_pred = svm.predict(X_test_std)
print('Accuracy: %.2f' % svm.score(X_test_std, Y_test))
\end{lstlisting}

\subsection{C 值网格搜索实现(plot\_c\_curve.py)}
为了找到最优的 $C$ 值,我们编写了 \texttt{plot\_c\_curve.py} 脚本进行系统的网格搜索:

\begin{lstlisting}[language=Python, caption=C 值网格搜索关键代码]
# 根据特征类型定义不同的 C 值搜索范围
if feature_type == 'attrs':
    C_values = np.logspace(-3, 2, 100)  # 属性特征: [0.001, 100]
elif feature_type == 'images':
    C_values = np.logspace(-6, 0, 100)  # 原始图片: [1e-6, 1]

# 对每个 C 值训练 SVM 并评估
accuracies = []
misclassified_counts = []
for C in C_values:
    svm = SVC(kernel="linear", C=C)
    svm.fit(X_train_std, Y_train)
    Y_pred = svm.predict(X_test_std)
    
    accuracy = svm.score(X_test_std, Y_test)
    misclassified = (Y_test != Y_pred).sum()
    
    accuracies.append(accuracy)
    misclassified_counts.append(misclassified)

# 找到最优 C 值
best_idx = np.argmax(accuracies)
best_C = C_values[best_idx]
best_accuracy = accuracies[best_idx]
\end{lstlisting}

该脚本会自动生成两张图表:错分样本数和准确率随 $C$ 值变化的曲线,帮助我们直观理解 $C$ 对模型性能的影响。

\vspace{1em}

% results summary table
\begin{table}[H]
    \centering
    \caption{不同特征与最优 C 值下的实验结果汇总}
    \begin{tabular}{lccc}
    \toprule
    \textbf{特征类型} & \textbf{特征维度} & \textbf{最优 C 值} & \textbf{测试准确率} \\
    \midrule
    属性特征 & 40 & $1.83 \times 10^{-2}$ & \textbf{92\%} \\
    原始像素 & 116,412 & $2.48 \times 10^{-5}$ & \textbf{64\%} \\
    \bottomrule
    \end{tabular}
\end{table}

\vspace{0.5em}

\section{实验结果}

\subsection{惩罚参数 C 的优化}

为了找到最佳的惩罚参数 $C$,我们在不同的 $C$ 值范围内进行了系统的网格搜索(对数尺度,共 100 个采样点)。两种特征类型的搜索范围和最优结果如表 1 所示。

\subsubsection{属性特征实验结果}

对于 40 维的属性特征,我们在 $C \in [10^{-3}, 10^{2}]$ 范围内搜索。结果显示最佳 $C = 1.83 \times 10^{-2}$,此时测试准确率达到 \textbf{92\%}(50 个测试样本中仅 4 个错分)。图 \ref{fig:terminal_attrs} 展示了使用最优 $C$ 值运行 \texttt{svm.py} 的终端输出。

\begin{figure}[H]
    \centering
    \fbox{\includegraphics[width=0.4\textwidth]{terminal_attrs.png}}
    \caption{属性特征在最优 C 值($1.83 \times 10^{-2}$)下的运行结果}
    \label{fig:terminal_attrs}
\end{figure}

\subsubsection{原始像素特征实验结果}

对于 116,412 维的原始像素特征,我们在 $C \in [10^{-6}, 1]$ 范围内搜索。结果显示最佳 $C = 2.48 \times 10^{-5}$,此时测试准确率为 \textbf{64\%}(50 个测试样本中 18 个错分)。图 \ref{fig:terminal_images} 展示了使用最优 $C$ 值运行 \texttt{svm.py} 的终端输出。

\begin{figure}[H]
    \centering
    \fbox{\includegraphics[width=0.4\textwidth]{terminal_images.png}}
    \caption{原始像素特征在最优 C 值($2.48 \times 10^{-5}$)下的运行结果}
    \label{fig:terminal_images}
\end{figure}

\subsection{C 值变化曲线分析}

图 \ref{fig:c_attrs} 和图 \ref{fig:c_images} 展示了测试准确率和错分样本数随 $C$ 值变化的完整趋势。从曲线可以清晰地观察到最优 $C$ 值的位置以及过拟合/欠拟合的区域。

\begin{figure}[H]
    \centering
    \includegraphics[width=\textwidth]{../log/c_value_curves_attrs.png}
    \caption{属性特征:准确率与错分样本数随 C 值的变化曲线}
    \label{fig:c_attrs}
\end{figure}

\begin{figure}[H]
    \centering
    \includegraphics[width=\textwidth]{../log/c_value_curves_images.png}
    \caption{原始像素特征:准确率与错分样本数随 C 值的变化曲线}
    \label{fig:c_images}
\end{figure}

\subsection{结果分析}

\subsubsection{软间隔与 C 值的理论分析}

在线性不可分的情况下,SVM 引入松弛变量 $\xi_i \geq 0$ 允许部分样本违反间隔约束,目标函数为:
\begin{equation}
\min_{w,b,\xi} \frac{1}{2}\|w\|^2 + C\sum_{i=1}^{n}\xi_i
\end{equation}

\noindent 其中 $C > 0$ 是惩罚参数,控制\textbf{间隔最大化}与\textbf{分类错误惩罚}之间的权衡:

\begin{itemize}
    \item \textbf{C 值较小}:更重视间隔最大化($\|w\|^2$ 项),允许更多训练样本被错分或落入间隔内,模型更平滑,泛化能力强,但可能\textbf{欠拟合}
    
    \item \textbf{C 值较大}:更重视减少训练误差($\sum\xi_i$ 项),强制模型正确分类更多训练样本,在训练集上表现好,但可能导致\textbf{过拟合}
\end{itemize}

\vspace{0.5em}
因此,$C$ 值的选择需要在偏差-方差之间取得平衡,这正是我们通过网格搜索寻找最优 $C$ 的原因。

\subsubsection{实验结果的具体分析}
\begin{enumerate}
    \item \textbf{属性特征表现更优}:尽管维度远低于原始像素(40 vs 116412),属性特征达到了 92\% 的准确率,显著优于原始像素的 64\%。这说明手工设计的属性特征(如面部特征点、纹理等)包含了更强的判别信息,而原始像素包含大量冗余。
    
    \item \textbf{最优 C 值差异显著}:
    \begin{itemize}
        \item 属性特征的最优 $C = 1.83 \times 10^{-2}$(较大):低维特征空间不易过拟合,可以使用较大的 $C$ 值来减少训练误差。
        \item 原始像素的最优 $C = 2.48 \times 10^{-5}$(很小):高维特征空间(116412 维)极易过拟合,必须使用很小的 $C$ 值施加强正则化,优先保证模型的泛化能力。
    \end{itemize}
    
    \item \textbf{C 值变化趋势}:
    \begin{itemize}
        \item 从图 \ref{fig:c_images} 可以看出,当 $C < 10^{-5}$ 时,模型欠拟合(准确率仅 30\%--40\%);
        \item 当 $C \approx 2.5 \times 10^{-5}$ 时达到最优(64\%);
        \item 当 $C > 10^{-4}$ 时,准确率略有下降,表明开始过拟合训练集。
    \end{itemize}
    
    \item \textbf{维度灾难}:原始像素的 116412 维特征空间远超训练样本数(250),导致严重的维度灾难。在这种情况下,SVM 需要极强的正则化(小 $C$)才能避免过拟合,但即使如此,准确率仍然受限。
\end{enumerate}

\section{改进方案与建议}

基于实验结果和分析,针对原始像素特征性能受限的问题,查阅资料后提出以下可行改进方案:

\begin{enumerate}
    \item \textbf{使用深度学习特征}:采用预训练卷积神经网络(如 ResNet18、VGG16)提取中间层特征(典型维度 512--2048),然后在这些语义化特征向量上训练 SVM。深度特征通常能显著提升准确率(预期可达 85\%+)。
    
    \item \textbf{PCA 降维}:使用主成分分析将原始像素从 116,412 维降至 200--1000 维,保留主要方差的同时去除噪声,加速训练并改善泛化。
    
    \item \textbf{尝试非线性核函数}:使用 RBF 核($K(x_i, x_j) = \exp(-\gamma \|x_i - x_j\|^2)$)捕捉非线性关系,同时对 $C$ 和 $\gamma$ 进行联合网格搜索。
    
    \item \textbf{数据增强}:对训练图像进行旋转、翻转、亮度调整等增强操作,扩充训练集规模,提升模型鲁棒性。
    
    \item \textbf{处理类别不均衡}:如果存在类别样本数差异,使用 \texttt{class\_weight='balanced'} 自动调整权重。
\end{enumerate}

\section{实验结论}

本次实验系统地研究了基于 SVM 的人脸图像分类问题,完成了以下工作:

\begin{enumerate}
    \item \textbf{实现了两种特征的 SVM 分类器}:基于属性特征(40 维)和原始像素特征(116,412 维)的线性 SVM,并通过标准化预处理提升了模型性能。
    
    \item \textbf{优化了惩罚参数 $C$}:通过对数尺度网格搜索(100 个采样点),找到了两种特征的最优 $C$ 值:
    \begin{itemize}
        \item 属性特征:$C = 1.83 \times 10^{-2}$,准确率 \textbf{92\%}
        \item 原始像素:$C = 2.48 \times 10^{-5}$,准确率 64\%
    \end{itemize}
    
    \item \textbf{分析了软间隔原理与 $C$ 值的关系}:验证了 $C$ 值控制偏差-方差权衡的理论,解释了不同特征下最优 $C$ 值差异巨大的原因(高维空间需要更强正则化)。
    
    \item \textbf{揭示了特征表示的重要性}:尽管属性特征维度远低于原始像素(40 vs 116,412),但其准确率显著更高(92\% vs 64\%),说明\textbf{特征工程比盲目使用高维原始特征更重要}。
\end{enumerate}

\vspace{0.5em}
\noindent 实验结果表明:在高维特征空间(如原始像素)中,样本数远少于特征维度时,模型容易陷入维度灾难。要突破 64\% 的准确率瓶颈,需要采用降维技术(PCA)或深度学习特征提取方法。

\end{document}
